% !TEX TS-program = pdflatex
% !TEX encoding = UTF-8 Unicode

% This is a simple template for a LaTeX document using the "article" class.
% See "book", "report", "letter" for other types of document.

\documentclass[10pt]{article} % use larger type; default would be 10pt

\usepackage[utf8]{inputenc} % set input encoding (not needed with XeLaTeX)

%%% Examples of Article customizations
% These packages are optional, depending whether you want the features they provide.
% See the LaTeX Companion or other references for full information.

%%% PAGE DIMENSIONS
\usepackage{geometry} % to change the page dimensions
\geometry{a4paper} % or letterpaper (US) or a5paper or....
% \geometry{margin=2in} % for example, change the margins to 2 inches all round
% \geometry{landscape} % set up the page for landscape
%   read geometry.pdf for detailed page layout information

\usepackage{indentfirst}

\usepackage{graphicx} % support the \includegraphics command and options

% \usepackage[parfill]{parskip} % Activate to begin paragraphs with an empty line rather than an indent

%%% PACKAGES
\usepackage{booktabs} % for much better looking tables
\usepackage{array} % for better arrays (eg matrices) in maths
\usepackage{paralist} % very flexible & customisable lists (eg. enumerate/itemize, etc.)
\usepackage{verbatim} % adds environment for commenting out blocks of text & for better verbatim
\usepackage{subfig} % make it possible to include more than one captioned figure/table in a single float
% These packages are all incorporated in the memoir class to one degree or another...

\usepackage{quoting}
\quotingsetup{vskip=3pt}

%%% HEADERS & FOOTERS
\usepackage{fancyhdr} % This should be set AFTER setting up the page geometry
\pagestyle{fancy} % options: empty , plain , fancy
\renewcommand{\headrulewidth}{0pt} % customise the layout...
\lhead{}\chead{}\rhead{}
\lfoot{}\cfoot{\thepage}\rfoot{}

%%% SECTION TITLE APPEARANCE
\usepackage{sectsty}
\allsectionsfont{\sffamily\mdseries\upshape} % (See the fntguide.pdf for font help)
% (This matches ConTeXt defaults)

%%% ToC (table of contents) APPEARANCE
\usepackage[nottoc,notlof,notlot]{tocbibind} % Put the bibliography in the ToC
\usepackage[titles,subfigure]{tocloft} % Alter the style of the Table of Contents
\renewcommand{\cftsecfont}{\rmfamily\mdseries\upshape}
\renewcommand{\cftsecpagefont}{\rmfamily\mdseries\upshape} % No bold!

%%% END Article customizations

%%% The "real" document content comes below...

\title{%
	Ransomware Analysis and Defense \\
	\large WanaCry and the Win32 environment}
\author{Justin Jones}
%\date{} % Activate to display a given date or no date (if empty),
         % otherwise the current date is printed 

\begin{document}
\maketitle

\tableofcontents

\section{Introduction}

A type of malware known as \emph{ransomware} has recently become very prevalent in the cyber security world, taking over user systems and demanding compensation for the safe return of all functionality and data.  While initially not incredibly sophisticated, this genre, if you will, has evolved from simple scripts that change file extensions and make empty threats to full-blown attacks affecting hundreds of thousands of systems worldwide that implement sophisticated NSA-developed exploits as their propogation vector. In this paper I will explore a specific piece of malware known as \emph{WanaCry} that recently made headlines around the world, performing a full static and dynamic analysis. I will then endeavor to write a useable piece of software to detect, stop and remove this malware, and then expand it to encompass any general, definable piece of software.

\section{Analysis}

\subsection{WanaCry/WCry}

\subsubsection{Background} 

WanaCry (referring to the general family consisting of all named variations of WannaCrypt, WCry, WanaCrypt, WanaCrypt0r, etc) came into prevalence during a massive attack starting on May 12, 2017. This software utilizes an exploit called EternalBlue\footnote{http://bit.ly/2spdT15}, a known vulnerability in the Server Message Block (SMB) protocol used by Microsoft Windows which was previously patched in a critical update outlined in KB4013389\footnote{https://support.microsoft.com/en-us/help/4013389/title}. As this vulnerability has been explored and detailed very thoroughly already, focus will be shifted to WanaCry's implementation and software aspects while avoiding the inner workings of the exploit. \par
The working sample of WanaCry has been obtained from theZoo\footnote{http://thezoo.morirt.com/}, with SHA256 hash: 
\begin{quoting} 
\centering
\emph{ed01ebfbc9eb5bbea545af4d01bf5f1071661840480439c6e5babe8e080e41aa}

\end{quoting}
and is positively identified by VirusTotal as a member of the WanaCry family\footnote{http://bit.ly/2s93pCl}.

\subsubsection{General File Data}

Utilizing PEiD\footnote{https://www.aldeid.com/wiki/PEiD}, it can be seen that the program was packed using Microsoft Visual Studio C++ 6.0 for Win32. 

Utilizing Dependency Walker\footnote{http://www.dependencywalker.com/}
\end{document}
